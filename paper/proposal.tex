\documentclass[11pt]{artikel3}
\usepackage{fullpage, setspace, graphicx}
\usepackage[margin=1in]{geometry}
\usepackage{times}


\title{UCD
\\MSc Thesis Proposal:\\\emph{Real-time System For Continuous Capacity Planning in Cloud Environment With Dynamic Pricing}}
\author{Kamil Smuga}
\date{\today}


\begin{document}
\maketitle


\section{Problem}

New IaaS solutions are available on the market with growing number of supported pricing models. This makes the offering attractive enough to move entire businesses to the Cloud. The availability of different pricing schemes reveals new challenges in already unpredictive Capacity Planning process. Is it better to pay some amount of money upfront and less hourly rate? Is is better to invest in less heavy utilized instances or maybe in more medium ones? When exactly one pricing scheme becomes more attractive? 

\begin{itemize}
  \item Thesis statement: An analysis of current utilization of the infrastructure in terms of price effectiveness in dynamic pricing IaaS model. A calculation of optimal configuration and adjustements based on recent data and future trends calculated from the load. 
  \item Project work: The system will accept incoming data from the IaaS provider with utilisation metrics and a number of running instances for a given type (light, medium, heavy, on demand). The data will be stored, analysed in terms of price optimality (subject for research) and graphed on capacity planning dashboard.
\end{itemize}


{\bf Motivation.} 
\begin{itemize}
  \item New startups often attempt to solve globe wide problems and have to either process increasing amounts of data either gather milions of users. Although the problem of scaling is hard enough by itself, a reality with limited budgets and business sustainability requires to tackle the scaling problem from the most optimal, cost-effective angle. Capacity planning can make a difference between making or losing the money in a dynamic, price variable environment. Based on a dynamic nature of bussiness size and external factors like pricing, the capacity planning needs to be continously adjusted and become a routine for daily operations. 
  \item Solving the problem of visualisation and optimisation of infrastructure costs will make decisions about pricing and provider choice possible in an educated and more  confident manner. 
\end{itemize}


{\bf Related Work.}
%Demonstrate the connection between your chosen problem and its foundations in existing work. 
\begin{itemize}
  \item What are the key theoretical models (e.g. process-based, formal language/complexity models, probability-based) and algorithms have been applied toward this problem previously? 
  \item What limitation and/or opportunity do you plan to address in your project/thesis?
  \item
In the related research literature, how is success measured (e.g. metrics and/or coverage of problem aspects)?
\end{itemize}


\section{Methodology}


\begin{itemize}
\item For projects, which libraries or software tools will be used for development, and at the highest level, what is the software design?
\item For theory-based projects and theses, what are the key theorems to be developed and/or proven? What proof techniques will be used?
\item For other theses, what algorithms will be adapted or devised, and what algorithms will they be compared with in your experiments?
\end{itemize}


\section{Evaluation}


\begin{itemize}
\item What data and software will be needed for your evaluation? 
\item What metrics
will you use to measure success? Commonly these include some subset of time, space, and accuracy (recognition rate, precision, recall, etc.).
Almost always, this should include reference to the evaluation methods described in the related work.
\item For empirical theses, an experiment is needed: which algorithms will be compared, and using what sets of parameters for each? If people are involved in the experiments, how will the experiment control for unwanted bias or confounds? {\bf How does the experiment test the hypothesis?}
\item {\bf How will you know when you are done?}
\end{itemize}


\section{Evaluation Outcomes}


\begin{itemize}
\item For your chosen assessment methods, what are the possible outcomes? 
\item Under which outcomes are the project goals achieved, or the hypothesis confirmed or rejected?
\end{itemize}


\bibliographystyle{plain}
\bibliography{preproposal}


**(Omitted) As an exercise, modify this document to include the references in the {\tt plain.bib} file.


\end{document}
