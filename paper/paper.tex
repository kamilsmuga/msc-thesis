\documentclass[]{final_report}
\usepackage{graphicx}
\usepackage{hyperref}
\usepackage[normalem]{ulem}

%%%%%%%%%%%%%%%%%%%%%%
%%% Project details
%%%%%%%%%%%%%%%%%%%%%%
\def\studentname{Kamil Smuga}
\def\projecttitle{An approach for Continuous Capacity Planning in Cloud Environments with Uptime-based Pricing Model}
\def\supervisorname{Liam Murphy}
\def\moderatorname{Christina Thorpe}

\begin{document}

\maketitle
\tableofcontents\pdfbookmark[0]{Table of Contents}{toc}\newpage

%%%%%%%%%%%%%%%%%%%%%%
%%% ABSTRACT 
%%%%%%%%%%%%%%%%%%%%%%

\begin{abstract}

\textsl{New Infrastructure as a Service solutions are becoming available with a growing number of supported pricing models. More often than not, a hosted Cloud environment is used to design and build an infrastructure for a product. The recent availability of different pricing schemes based on resource utilization and uptime reveals new challenges in already unpredictable capacity planning process. There is a choice between ad-hoc provisioning and upfront payments with reduced hourly rates. Reserved instances charged upfront are categorized into three groups: light, medium and heavy. Which one is better for a given utilization model? When exactly does one pricing scheme becomes more cost effective? Determining which machine type is better for a given utilization model, or at which point the cost effectiveness of a pricing scheme changes, is vital for the companies subscribing to the IaaS. }

\end{abstract}
\newpage

%%%%%%%%%%%%%%%%%%%%%%
%%% INTRODUCTION 
%%%%%%%%%%%%%%%%%%%%%%

\chapter{Introduction}

\section{Objectives}

\section{Structure of the Document}

\newpage

%%%%%%%%%%%%%%%%%%%%%%
%%% BACKGROUND 
%%%%%%%%%%%%%%%%%%%%%%

\chapter{Background}

\section{The Problem Domain}

\section{Motivation}

\section{IaaS consumer point of view}

\section{IaaS provider uptime-based pricing schemes}

\section{Literature review}


%%%%%%%%%%%%%%%%%%%%%%
%%% OPTIMALITY 
%%% ALGORITHM 
%%%%%%%%%%%%%%%%%%%%%%

\chapter{Design and Implementation of Optimality Algorithm Based on Uptime and Pricing Scheme}

\section{Design}

\section{Implementation}

%%%%%%%%%%%%%%%%%%%%%%  
%%% ALGORITHM TESTING  
%%% ON GOOGLE DATASET 
%%%%%%%%%%%%%%%%%%%%%%

\chapter{Algorithm Testing on Google Cluster Usage Dataset}

\section{Calculate cost per hour for each pricing scheme: light, medium and heavy}

\section{Graph price per hour on y and number of hours/day on x to find cut points}

\section{Calculate optimality factor based on distance from cut points}

%%%%%%%%%%%%%%%%%%%%%%
%%% CAPACITY PLANNING 
%%% STRATEGIES    
%%%%%%%%%%%%%%%%%%%%%%

\chapter{Capacity Planning Strategies}

\section{Peak-to-mean ratio based resource assignment}

\section{Only heavy instances}

%%%%%%%%%%%%%%%%%%%%%%
%%% CONCLUSION 
%%%%%%%%%%%%%%%%%%%%%%

\chapter{Conclusion and Further Work}

%%%%%%%%%%%%%%%%%%%%%%
%%% BIBLIOGRAPHY 
%%%%%%%%%%%%%%%%%%%%%%

\newpage
\begin{thebibliography}{99}
\bibitem{DAWSON:2000} Christian Dawson. \emph{The Essence of Computing Projects -- A Student's Guide}. 192 pages. ISBN: 013021972X. Pearson Education, 2000.
\end{thebibliography}
\label{endpage}

\end{document}

\end{article}